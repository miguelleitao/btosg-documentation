\href{https://travis-ci.org/miguelleitao/btosg}{\tt } \section*{btosg}

A thin abstraction layer to integrate {\bfseries Bullet} and {\bfseries Open\+Scene\+Graph}.

\subsubsection*{Description}

{\bfseries btosg} is aimed to ease building simple visual simulation applications integrating {\bfseries Bullet} and {\bfseries Open\+Scene\+Graph}. {\bfseries btosg} stands on top of these two A\+P\+Is but does not try to hide them. Instead, in order to build a complete application, the programmer is able and usually needs to access directly the data structures from both {\bfseries Bullet} and {\bfseries Open\+Scene\+Graph}. So, the use of {\bfseries btosg} does not avoid the requirement to know and understand their A\+P\+Is.

The use of {\bfseries btosg} can help the programming task because it allows to create and position both the physical and graphics definitions of small objects in one step. It also keeps track of syncronizing the graphics definitions from the state of each physical object.

\subsubsection*{Dependences}

\paragraph*{Open\+Scene\+Graph\+:}


\begin{DoxyItemize}
\item \href{https://github.com/openscenegraph/OpenSceneGraph}{\tt https\+://github.\+com/openscenegraph/\+Open\+Scene\+Graph}
\item \href{http://www.openscenegraph.org/}{\tt http\+://www.\+openscenegraph.\+org/} \paragraph*{Bullet\+:}
\end{DoxyItemize}


\begin{DoxyItemize}
\item \href{https://github.com/bulletphysics/bullet3}{\tt https\+://github.\+com/bulletphysics/bullet3}
\item \href{http://bulletphysics.org/}{\tt http\+://bulletphysics.\+org/}
\end{DoxyItemize}

Dependences can be installed by (assuming Fedora Linux)\+: dnf install bullet openscenegraph-\/osg openscenegraph-\/osg\+Viewer openscenegraph-\/osg\+Sim openscenegraph-\/osg\+DB openscenegraph-\/osg\+GA openscenegraph-\/osg\+Shadow

\subsubsection*{Build}

git clone \href{https://github.com/miguelleitao/btosg.git}{\tt https\+://github.\+com/miguelleitao/btosg.\+git} cd btosg make sudo make install

\subsubsection*{Usage}

Look at provided examples. Prepare your {\itshape application.\+cpp} using \begin{DoxyVerb}#include <btosg.h>
\end{DoxyVerb}


Compile using 
\begin{DoxyPre}
g++ -c `pkg-config --cflags btosg` {\itshape application.cpp}
g++ -o {\itshape application} `pkg-config --libs btosg` {\itshape application.o}
\end{DoxyPre}
 \subsubsection*{Examples}

{\bfseries btosg} is available with some working examples.
\begin{DoxyItemize}
\item {\bfseries ball.\+cpp} implements a simple simulation of a ball with two planes.
\item {\bfseries objects.\+cpp} provides an example for creating a complete object (graphical and physical) from loading an external Wavefront O\+BJ file.
\item {\bfseries car.\+cpp} implements a basic vehicle with four wheels and suspensions. It can be compiled using a Z or Y pointing up vector. Usage instructions are provided in source file.
\end{DoxyItemize}

To compile and try the provided examples do\+: \begin{DoxyVerb}make examples 
./ball
./objects
./carZ
./carY\end{DoxyVerb}
 