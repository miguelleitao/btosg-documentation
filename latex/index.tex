A programming framework to build visual physical simulations.

btosg is aimed to ease building simple visual physical simulation applications integrating Bullet and Open\+Scene\+Graph. btosg stands on top of these two A\+P\+Is but does not try to hide them. Instead, in order to build a complete application, the programmer is able and usually needs to access directly the data structures from both Bullet and Open\+Scene\+Graph. So, the use of btosg does not avoid the requirement to know and understand their A\+P\+Is.

The use of btosg can help the programming task because it allows to create and position both the physical and graphics definitions of small objects in one step. It also keeps track of syncronizing the graphics definitions from the state of each physical object.\hypertarget{index_contents}{}\section{Contents}\label{index_contents}
Contents

Dependences Building btosg Usage Examples

htmlinc  simples 
\begin{DoxyCodeInclude}
/*!
\(\backslash\)mainpage btosg

\(\backslash\)section intro
*/

A thin abstraction layer to integrate **Bullet** and **OpenSceneGraph**.

### Description 
**btosg** is aimed to ease building simple visual simulation applications integrating **Bullet** and
       **OpenSceneGraph**.
**btosg** stands on top of these two APIs but does not try to hide them. 
Instead, in order to build a complete application, the programmer is able and usually needs to access
       directly the data structures from both **Bullet** and **OpenSceneGraph**. So, the use of **btosg** does not avoid
       the requirement to know and understand their APIs.

The use of **btosg** can help the programming task because it allows to create and position both the
       physical and graphics definitions of small objects in one step. It also keeps track of syncronizing the graphics
       definitions from the state of each physical object.

### Dependences
#### OpenSceneGraph: 
* https://github.com/openscenegraph/OpenSceneGraph 
* http://www.openscenegraph.org/

#### Bullet:
* https://github.com/bulletphysics/bullet3 
* http://bulletphysics.org/

Dependences can be installed on Fedora Linux using:

    dnf install bullet bullet-devel OpenSceneGraph OpenSceneGraph-devel

Dependences can be installed on Ubuntu using:

    apt install bullet openscenegraph-osg openscenegraph-osgViewer openscenegraph-osgSim
       openscenegraph-osgDB openscenegraph-osgGA openscenegraph-osgShadow

### Build

    git clone https://github.com/miguelleitao/btosg.git
    cd btosg
    make
    sudo make install

### Usage
Look at provided examples. Prepare your \_application.cpp\_ using

    #include <btosg.h>

Compile using
<pre>
g++ -c `pkg-config --cflags btosg` <i>application.cpp</i>
g++ -o <i>application</i> `pkg-config --libs btosg` <i>application.o</i>
</pre>
### Examples
**btosg** is available with some working examples.
* **ball.cpp** implements a simple simulation of a ball with two planes.
* **objects.cpp** provides an example for creating a complete object (graphical and physical) from loading
       an external Wavefront OBJ file. 
* **car.cpp** implements a basic vehicle with four wheels and suspensions. It can be compiled using a Z or
       Y pointing up vector.
Usage instructions are provided in source file.

To compile and try the provided examples do:

    make examples 
    ./ball
    ./objects
    ./carZ
    ./carY

\end{DoxyCodeInclude}
 fim 